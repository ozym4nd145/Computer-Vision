\documentclass{article}
\usepackage{graphicx}
\usepackage{geometry}
\usepackage{amsmath,mathpazo}
\usepackage{amssymb}
\usepackage{mathtools}
\usepackage{commath}
\usepackage{enumitem}

\geometry{
  top=20mm,
}
\newcommand{\boldvec}[1]{\boldsymbol{\vec{\textbf{#1}}}}
\newcommand{\capvec}[1]{\boldsymbol{\hat{\textbf{#1}}}}
\begin{document}

\title{COL780 - Computer Vision\\Minor 1}
\author{Suyash Agrawal\\2015CS10262}

\maketitle

\section*{Question 5}
Expanding given appromixation equation we get:
\begin{align*}
u &= u_0 + xu_x + yu_y\\
v &= v_0 + xv_x + yv_y
\end{align*}
The Optical flow equation is:
$$ uI_x + vI_y + I_t = 0 $$
Putting the given equation in optical flow equation, we get:
$$ I_x(u_0 + xu_x + yu_y) + I_y(v_0 + xv_x + yv_y) = -I_t $$
Now, we take a region around the point $(x,y)$ and we assume that optical flow assumptions
hold in this region.\\
Thus, using the above equation, we can transform our case into a least square problem as follows:\\
Define matrix $\mathbf{A}$ as:
$$
\mathbf{A} = \begin{bmatrix}
  I_x(p1) & x_{p1}I_x(p1) & y_{p1}I_x(p1) & I_y(p1) & x_{p1}I_y(p1) & y_{p1}I_y(p1)\\
  I_x(p2) & x_{p2}I_x(p2) & y_{p2}I_x(p2) & I_y(p2) & x_{p2}I_y(p2) & y_{p2}I_y(p2)\\
  \vdots & \vdots & \vdots & \vdots & \vdots & \vdots \\
  I_x(pn) & x_{pn}I_x(pn) & y_{pn}I_x(pn) & I_y(pn) & x_{pn}I_y(pn) & y_{pn}I_y(pn)
\end{bmatrix}
$$
and $\vec{u}$ and $\vec{b}$ as:
$$
\vec{u} = \begin{bmatrix}
  u_0 \\
  u_x \\
  u_y \\
  v_0 \\
  v_x \\
  v_y
\end{bmatrix}
\qquad
\vec{b} = -\begin{bmatrix}
  I_t(p1) \\
  I_t(p2) \\
  \vdots \\
  I_t(pn)
\end{bmatrix}
$$
and our least square problem becomes
$$ \min \quad \lVert \mathbf{A}\vec{u} - \vec{b} \rVert^2 $$
whose solution comes out to be:
$$ \vec{u} = (A^TA)^{-1}A^Tb$$
The assumption that we made in this case are:
\begin{enumerate}
\item
The values of the parameter don't change for the $n$ points near $(x,y)$.
\item
Matrix $A^TA$ is invertible.
\end{enumerate}
\pagebreak

\section*{Question 6}
The velocity of the object is:
$$ V = -U-\Omega \times P $$
which, when expanded, translates to:
\begin{align*}
V_1 &= \frac{dX}{dt} = -U_1 - \Omega_2Z + \Omega_3Y \\
V_2 &= \frac{dY}{dt} = -U_2 - \Omega_3Z + \Omega_1Z \\
V_3 &= \frac{dZ}{dt} = -U_3 - \Omega_1Y + \Omega_2X 
\end{align*}
Also, it is given that:
$$
x=\frac{fX}{Z} \quad y = \frac{fY}{Z}
$$
and by vector calculus:
$$
u = \frac{dx}{dt} \qquad v = \frac{dy}{dt}
$$
and by putting value of $x$ from above:
\begin{align*}
  u &= \frac{d(\frac{fX}{Z})}{dt} \\
    &= f\frac{Z\frac{dX}{dt} - X\frac{dZ}{dt}}{Z^2} \\
    &= f\frac{ZV_1 - XV_3}{Z^2} \\
    &= \frac{f}{Z}(-U_1 - \Omega_2Z + \Omega_3Y) - \frac{x}{Z}(-U_3 - \Omega_1Y + \Omega_2X)\\
    &= -U_1\frac{f}{Z} - \Omega_2f + \Omega_3y - x (-\frac{U_3}{Z} - \frac{\Omega_1y}{f}+\frac{\Omega_2x}{f})
\end{align*}
Therefore,\\
$$u = -U_1\frac{f}{Z} - \Omega_2f + \Omega_3y - x (-\frac{U_3}{Z} - \frac{\Omega_1y}{f}+\frac{\Omega_2x}{f})$$
Similary, putting value of $y$, we get:
\begin{align*}
  v &= \frac{d(\frac{fY}{Z})}{dt} \\
    &= f\frac{Z\frac{dY}{dt} - Y\frac{dZ}{dt}}{Z^2} \\
    &= f\frac{ZV_2 - YV_3}{Z^2} \\
    &= \frac{f}{Z}(-U_2 - \Omega_3Z + \Omega_1Z) - \frac{y}{Z}(-U_3 - \Omega_1Y + \Omega_2X)\\
    &= -U_2\frac{f}{Z} - \Omega_3x + \Omega_1f - y (-\frac{U_3}{Z} - \frac{\Omega_1y}{f}+\frac{\Omega_2x}{f})
\end{align*}
Therefore,\\
$$v = -U_2\frac{f}{Z} - \Omega_3x + \Omega_1f - y (-\frac{U_3}{Z} - \frac{\Omega_1y}{f}+\frac{\Omega_2x}{f})$$
Hence Proved.

\pagebreak
\section*{Question 7}
From previous question, it is given that:
\begin{align*}
u &= -U_1\frac{f}{Z} - \Omega_2f + \Omega_3y - x (-\frac{U_3}{Z} - \frac{\Omega_1y}{f}+\frac{\Omega_2x}{f}) \\
v &= -U_2\frac{f}{Z} - \Omega_3x + \Omega_1f - y (-\frac{U_3}{Z} - \frac{\Omega_1y}{f}+\frac{\Omega_2x}{f})
\end{align*}
Now, since we are taking first order linear approximation of $u$ and $y$:
\begin{align*}
u &= -U_1\frac{f}{Z} - \Omega_2f + \Omega_3y + x\frac{U_3}{Z} +O(x^2,xy,y^2) \\
v &= -U_2\frac{f}{Z} + \Omega_1f - \Omega_3x + y\frac{U_3}{Z} +O(x^2,xy,y^2)
\end{align*}
After taking partial derivative of $u$ w.r.t. $x$ we get:
\begin{align*}
\frac{\partial{u}}{\partial{x}} &= U_1f\frac{\frac{\partial{Z}}{\partial{x}}}{Z^2}
                                + 0 + 0 + \frac{U_3}{Z} - xU_3\frac{\frac{\partial{Z}}{\partial{x}}}{Z^2} \\
                                &= U_1f\frac{Z_x}{Z^2} + \frac{U_3}{Z} + \textit{(tending to 0 terms)} \\
\Aboxed{\therefore u_x &= U_1f\frac{Z_x}{Z^2} + \frac{U_3}{Z}}\\
\end{align*}
Similarly, for others:
\begin{align*}
\frac{\partial{u}}{\partial{y}} &= U_1f\frac{\frac{\partial{Z}}{\partial{y}}}{Z^2}
                                + 0 + \Omega_3 + 0 - xU_3\frac{\frac{\partial{Z}}{\partial{y}}}{Z^2} \\
                                &= U_1f\frac{Z_y}{Z^2} + \Omega_3 + \textit{(tending to 0 terms)} \\
\Aboxed{\therefore u_y &= U_1f\frac{Z_y}{Z^2} + \Omega_3}\\
\frac{\partial{v}}{\partial{x}} &= U_2f\frac{\frac{\partial{Z}}{\partial{x}}}{Z^2}
                                + 0 - \Omega_3 + 0 - yU_3\frac{\frac{\partial{Z}}{\partial{x}}}{Z^2} \\
                                &= U_2f\frac{Z_x}{Z^2} - \Omega_3 + \textit{(tending to 0 terms)} \\
\Aboxed{\therefore v_x &= U_2f\frac{Z_x}{Z^2} - \Omega_3}\\
\frac{\partial{v}}{\partial{y}} &= U_2f\frac{\frac{\partial{Z}}{\partial{y}}}{Z^2}
                                + 0 + 0 + \frac{U_3}{Z} - yU_3\frac{\frac{\partial{Z}}{\partial{y}}}{Z^2} \\
                                &= U_2f\frac{Z_y}{Z^2} + \frac{U_3}{Z} + \textit{(tending to 0 terms)} \\
\Aboxed{\therefore v_y &= U_2f\frac{Z_y}{Z^2} + \frac{U_3}{Z} }\\
\end{align*}

Given:
\begin{align*}
  V_z = \frac{U_3}{Z}, \quad V_x = \frac{U_1}{Z}, \quad V_y = \frac{U_2}{Z},
  \quad Z_X = \frac{fZ_x}{Z} \quad\textit{and}\quad Z_Y = \frac{fZ_y}{Z}
\end{align*}
We get:
\begin{alignat*}{3}
u_x \quad &= \quad U_1f\frac{Z_x}{Z^2} + \frac{U_3}{Z} \quad &&= \quad V_z + V_xZ_X\\
u_y \quad &= \quad U_1f\frac{Z_y}{Z^2} + \Omega_3 \quad &&= \quad \Omega_3 + V_xZ_Y\\
v_x \quad &= \quad U_2f\frac{Z_x}{Z^2} - \Omega_3 \quad &&= \quad -\Omega_3 + V_yZ_X\\
v_y \quad &= \quad U_2f\frac{Z_y}{Z^2} + \frac{U_3}{Z} \quad &&= \quad V_z + V_yZ_Y\\
\end{alignat*}
Hence Proved.
\section*{Question 8}
Given statements:
\begin{align*}
  div\mathbf{v} &= (u_x+v_y) \\
  curl\mathbf{v} &= -(u_y-v_x) \\
  (def\mathbf{v})cos2\mu &= (u_x - v_y) \\ 
  (def\mathbf{v})sin2\mu &= (u_y + v_x) \\ 
\end{align*}
Putting values in the equation:
$$
  \frac{div\mathbf{v}}{2}\begin{bmatrix}1&0\\0&1\end{bmatrix}
+ \frac{curl\mathbf{v}}{2}\begin{bmatrix}0&-1\\1&0\end{bmatrix}
+ \frac{def\mathbf{v}}{2}\begin{bmatrix} cos2\mu & sin2\mu \\ sin2\mu & -cos2\mu \end{bmatrix}
$$
we get :
\begin{align*}
  &= \frac{(u_x+v_y)}{2}\begin{bmatrix}1&0\\0&1\end{bmatrix}
+ \frac{(v_x - u_y)}{2}\begin{bmatrix}0&-1\\1&0\end{bmatrix}
+ \frac{def\mathbf{v}}{2}\begin{bmatrix} cos2\mu & sin2\mu \\ sin2\mu & -cos2\mu \end{bmatrix}
\\
  &= \begin{bmatrix}\frac{(u_x+v_y)}{2}&0\\0&\frac{(u_x+v_y)}{2}\end{bmatrix}
+ \begin{bmatrix}0&\frac{(u_y-v_x)}{2}\\\frac{(v_x - u_y)}{2}&0\end{bmatrix}
+ \begin{bmatrix}(\frac{def\mathbf{v}}{2})cos2\mu &(\frac{def\mathbf{v}}{2})sin2\mu \\
                  (\frac{def\mathbf{v}}{2})sin2\mu & -(\frac{def\mathbf{v}}{2})cos2\mu \end{bmatrix}
\\
  &= \begin{bmatrix} \frac{(u_x+v_y)}{2} &\frac{(u_y-v_x)}{2} \\
                  \frac{(u_y-v_x)}{2} &\frac{(v_y+u_x)}{2}\end{bmatrix}
+ \begin{bmatrix} \frac{(u_x-v_y)}{2} &\frac{(u_y+v_x)}{2} \\
                  \frac{(u_y+v_x)}{2} &\frac{(v_y-u_x)}{2}\end{bmatrix}
\\
 &= \begin{bmatrix} u_x & u_y \\ v_x & v_y \end{bmatrix}
\end{align*}
Therefore,
$$
\begin{bmatrix} u_x & u_y \\ v_x & v_y \end{bmatrix}
= \frac{div\mathbf{v}}{2}\begin{bmatrix}1&0\\0&1\end{bmatrix}
+ \frac{curl\mathbf{v}}{2}\begin{bmatrix}0&-1\\1&0\end{bmatrix}
+ \frac{def\mathbf{v}}{2}\begin{bmatrix} cos2\mu & sin2\mu \\ sin2\mu & -cos2\mu \end{bmatrix}
$$
Hence Verified.
\pagebreak

\section*{Question 9}
We know that $\boldsymbol{\hat{\textbf{Q}}}$ is the unit vector in the camera look-at direction. Therefore,
$$ \boldsymbol{\hat{\textbf{Q}}} = (0,0,1)$$
Also, $\boldvec{U}$ is the translational velocity of the object. Therefore,
\begin{align*}
    \boldvec{U} &= (U_1,U_2,U_3)\\
    \boldvec{A} &= \frac{\boldvec{U} - (\boldvec{U} \cdot \capvec{Q})\capvec{Q}}{Z} \\
                &= \frac{(U_1,U_2,U_3) - (0,0,U_3)}{Z} \\
    \therefore \boldvec{A} &= \frac{(U_1,U_2,0)}{Z}
\end{align*}
Next, we write $\nabla Z$ as:
\begin{align*}
\nabla Z &= (\frac{\partial{Z}}{\partial{x}},\frac{\partial{Z}}{\partial{y}},0)
        = (Z_x,Z_y,0) \\
\therefore \boldvec{F}&= \frac{f}{Z}\nabla Z = \frac{f}{Z}(Z_x,Z_y,0)
\end{align*}
Also, from \textit{Question 8} we have:
\begin{align*}
  div\mathbf{v} &= (u_x+v_y) \\
  curl\mathbf{v} &= -(u_y-v_x) \\
  (def\mathbf{v})cos2\mu &= (u_x - v_y) \\ 
  (def\mathbf{v})sin2\mu &= (u_y + v_x) \\ 
\end{align*}
And from \textit{Question 7} we have:
\begin{align*}
u_x &= U_1f\frac{Z_x}{Z^2} + \frac{U_3}{Z}\\
u_y &= U_1f\frac{Z_y}{Z^2} + \Omega_3 \\
v_x &= U_2f\frac{Z_x}{Z^2} - \Omega_3 \\
v_y &= U_2f\frac{Z_y}{Z^2} + \frac{U_3}{Z}
\end{align*}
Putting the values, we get:
\begin{align*}
    div\mathbf{v} &= 2\frac{U_3}{Z} + \frac{f}{Z^2}(U_1Z_x + U_2Z_y)\\
    curl\mathbf{v} &= -2\Omega_3 + \frac{f}{Z^2}(U_2Z_x - U_1Z_y)\\
    (def\mathbf{v})cos2\mu &= \frac{f}{Z^2}(U_1Z_x - U_2Z_y)\\
    (def\mathbf{v})sin2\mu &= \frac{f}{Z^2}(U_1Z_y + U_2Z_x)\\ 
\end{align*}
Now let us evaluate:
\begin{align*}
    & 2\frac{\boldvec{U}\cdot\capvec{Q}}{Z} + \boldvec{F}\cdot\boldvec{A}\\
    =& 2\frac{(U_1,U_2,U_3)\cdot(0,0,1)}{Z} + \frac{f}{Z}(Z_x,Z_y,0)\cdot \frac{(U_1,U_2,0)}{Z} \\
    =& 2\frac{U_3}{Z} + \frac{f}{Z^2}(Z_xU_1 + Z_yU_2)\\
    =& div\mathbf{v}\\
    \Aboxed{\therefore div\mathbf{v} =& 2\frac{\boldvec{U}\cdot\capvec{Q}}{Z} + \boldvec{F}\cdot\boldvec{A}}\\
\end{align*}
Similarly, for others:
    % -2\boldvec{\Omega} \cdot \capvec{Q} + \abs{ (\boldvec{F} \times \boldvec{A}) }
\begin{align*}
    &-2\boldvec{$\Omega$} \cdot \capvec{Q} + \abs{ (\boldvec{F} \times \boldvec{A}) }\\
    =& -2(\Omega_1,\Omega_2,\Omega_3)\cdot(0,0,1) + \abs{\frac{f}{Z}(Z_x,Z_y,0)\times \frac{(U_1,U_2,0)}{Z}}\\
    =& -2\Omega_3 + \abs{\frac{f}{Z^2}(0,0,U_2Z_x - U_1Z_y)}\\
    =& -2\Omega_3 + \frac{f}{Z^2}(U_2Z_x - U_1Z_y)\\
    =& curl\mathbf{v}\\
    \Aboxed{\therefore curl\mathbf{v} =& -2\boldvec{$\Omega$} \cdot \capvec{Q} + \abs{ (\boldvec{F} \times \boldvec{A}) }}
\end{align*}

Now :
\begin{align*}
  def\mathbf{v} &= \sqrt{((def\mathbf{v})cos2\mu)^2 + ((def\mathbf{v})sin2\mu)^2}\\
                &= \frac{f}{Z^2}\sqrt{ U_1^2Z_x^2 + U_2^2Z_y^2 + U_2^2Z_x^2 + U_1^2Z_y^2 }\\
                &= \frac{f}{Z^2}\sqrt{U_1^2 + U_2^2}\sqrt{Z_x^2 + Z_y^2}\\
                &= \abs{\frac{(U_1,U_2,0)}{Z}} + \abs{\frac{f}{Z}(Z_x,Z_y,0)}\\
                &= \abs{\boldvec{F}}\abs{\boldvec{A}} \\
  \Aboxed{\therefore def\mathbf{v} &= \abs{\boldvec{F}}\abs{\boldvec{A}} }
\end{align*}
Finally,
\begin{align*}
tan2\mu &= \frac{(def\mathbf{v})sin2\mu}{(def\mathbf{v})cos2\mu}\\
        &= \frac{\frac{f}{Z^2}(U_1Z_y + U_2Z_x)}{\frac{f}{Z^2}(U_1Z_x - U_2Z_y)}\\
        &= \frac{(U_1Z_y + U_2Z_x)}{(U_1Z_x - U_2Z_y)}\\
        &= \frac{(\frac{Z_y}{Z_x} + \frac{U_2}{U_1})}{(1 - \frac{U_2}{U_1}\frac{Z_y}{Z_x})}\\
\end{align*}
And we already have:
\begin{align*}
  tan(\angle A) = \frac{U_2}{U_1} \qquad tan(\angle F) = \frac{Z_y}{Z_x}
\end{align*}
Therefore,
\begin{align*}
  tan2\mu &= \frac{tan(\angle A)+tan(\angle F)}{1 - tan(\angle A)tan(\angle F)}\\
          &= \tan(\angle A + \angle F)\\
  2\mu &= \angle A + \angle F \\
  \Aboxed{\therefore \mu &= \frac{(\angle A + \angle F)}{2}}
\end{align*}
Hence Proved.

\section*{Question 10}
\begin{enumerate}[label=(\alph*)]

\item Deformation can be used to encode the surface orientation as it gives us an idea of how object deforms
with small change in object position, thus revealing information about the orientation. Similarly, divergence
is a measure of how objects scale as they move towards camera, and thus high divergence implies that velocity
is high and thus time of collision will be shorter. Also, these components are unaffected by viewer rotations
such as panning or tilting of the camera unlike point image velocities, which change considerably.

\item If $\boldvec{A}=0$ then $div\mathbf{v} = 2\frac{\boldvec{U}\cdot\capvec{Q}}{Z}$. Also,
$\boldvec{U}\cdot\capvec{Q}$ encodes the velocity of object in direction of camera and $Z$ is distance of
object. Thus, time of collision is $\frac{Z}{\boldvec{U}\cdot\capvec{Q}}$ which is $\frac{2}{div\mathbf{v}}$.
Thus, if $\boldvec{A}=0$ then we can determine time of collision.\\
Also, in general motion, the time of collision is bounded by:
$$
\frac{2}{div\mathbf{v} + def\mathbf{v}} \leq \mathbf{t_c} \leq \frac{2}{div\mathbf{v} - def\mathbf{v}}
$$

\item It can be seen that shallow objects that are near to camera will produce same affect as deep structures
which are far away from camera, because in pinhole model the motion registered are actually scaled components of
actual motion ($x = f\frac{X}{Z}$) and thus a large motion which is far away will approximate to small motion that
is near (as their ratio will be of same order).

\item I think that egomotion can be used to gain some knowledge about the surface orientation, assuming that
the object in question is actually constant. This is because small changes in the direction of camera look-at direction
can be used to map the surface gradient of the object, and thus can give us an idea of the orientation of surface of the object.

\end{enumerate}

\section*{Declaration of Originality}
I hereby claim that the work presented here is my own and not copied from anywhere. Though, while writing these
answers I had discussions with Aman Agrawal (2015CS10210) , Ankesh Gupta (2015CS10435) and Saket Dingliwal (2015CS10254).\\
These discussions were mainly focused on clearing doubts on the meaning of question and clearing diagramatic representation
of the given situation. Also, I referred to a paper cited in the References for help in understanding the context and
background of question 6.\\

\begin{thebibliography}{1}
  \bibitem{imagediv} 
  Roberto Cipolla and Andrew Blake.
  \textit{Image divergence and deformation from closed curves}.\\ 
  Int. Journal of Robotics Research, 16(1):77-96, 1997.
\end{thebibliography}

\end{document}
